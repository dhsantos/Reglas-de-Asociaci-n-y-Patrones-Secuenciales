\documentclass{endm}
\usepackage{endmmacro}
\usepackage{graphicx}
\usepackage{float}
\usepackage[utf8]{inputenc}
\usepackage[spanish,activeacute]{babel}
\usepackage{setspace}  
\setlength{\textwidth}{150mm}

% The following is enclosed to allow easy detection of differences in
% ascii coding.
% Upper-case    A B C D E F G H I J K L M N O P Q R S T U V W X Y Z
% Lower-case    a b c d e f g h i j k l m n o p q r s t u v w x y z
% Digits        0 1 2 3 4 5 6 7 8 9
% Exclamation   !           Double quote "          Hash (number) #
% Dollar        $           Percent      %          Ampersand     &
% Acute accent  '           Left paren   (          Right paren   )
% Asterisk      *           Plus         +          Comma         ,
% Minus         -           Point        .          Solidus       /
% Colon         :           Semicolon    ;          Less than     <
% Equals        =           Greater than >          Question mark ?
% At            @           Left bracket [          Backslash     \
% Right bracket ]           Circumflex   ^          Underscore    _
% Grave accent  `           Left brace   {          Vertical bar  |
% Right brace   }           Tilde        ~

\newcommand{\Nat}{{\mathbb N}}
\newcommand{\Real}{{\mathbb R}}
\begin{document}
% DO NOT REMOVE: Creates space for Elsevier logo, ScienceDirect logo
% and ENDM logo
\begin{verbatim}\end{verbatim}\vspace{.5cm}
\begin{frontmatter}
\title{Las Reglas del Congreso}
\author{Diego Santos\thanksref{sdemail}}
\author{Axel Savizky  \thanksref{raemail}}
\address{Reglas de Asociación y Simulación de Patrones - Departamento de Computación - Universidad de Buenos Aires - Bs.As., Argentina}
\thanks[raemail]{Email:
  \href{mailto:axel.savizky@gmail.com} {\texttt{\normalshape
   axel.savizky@gmail.com}}}
\thanks[sdemail]{Email:
  \href{mailto:diego.h.santos@gmail.com} {\texttt{\normalshape
   diego.h.santos@gmail.com}}}
\begin{abstract}
Todos los años durante el período de marzo a diciembre 257 diputados se reunen para determinar que nuevas leyes regirán sobre el pueblo argentino. En el presente trabajo se analizó un conjunto de votaciones de la Cámara de Diputados de la Nación con el fin de identificar las reglas que se esconden en las sanciones de leyes.
\end{abstract}
\begin{keyword}
Diputados, Transacciones, Monotonía, Leyes, Ausencias.
\end{keyword}
\end{frontmatter}
\section{Presentación}\label{intro}
El objetivo del trabajo es determinar reglas en las votaciones de ley de la Cámara de Diputados del Congreso de la Nación Argentina a partir de los resultados de las votaciones de cada ley correspondientes al período de marzo a septiembre de 2018.\\

Durante este período se trataron 72 leyes, involucrando a cada una 257 diputados de 32 partidos poltíticos agrupados en 17 interbloques.\\

La datos para la confección del reporte se obtuvieron de la página web de la Cámara de Diputados. \footnote{https://www.diputados.gov.ar/}

\subsection{Set de Datos}

El set de datos utilizado consta de 72 archivos csv donde en cada uno de ellos se muestra por fila: \\

Diputado $|$ Partido $|$ Provincia $|$ ¿Cómo voto?\\

Es decir que cada archivo tiene 257 filas donde en cada una se presenta un diputado, el partido al que pertenece, la provincia a la que representa y cual fue su voto.\\

El campo ¿Cómo voto? tiene 5 estados posibles:\\

\begin{itemize}
\item Presidente: Si ejerció como presidente de la sesión.
\item Afirmativo: Si su voto a favor de la sanción de la ley.
\item Negativo: Si su voto fue en contra de la sanción de la ley.
\item Abstención: Si se excuso de votar.
\item Ausente: Si no estuvo presente en el recinto.
\end{itemize}

\subsection{Transacción}

Los datos en el formato que se obtuvieron no resultan útiles para establecer reglas más allá de las evidentes, por lo que fue necesario reagrupar los datos en otro tipo de transacción. \\

De un análisis de los datos en crudo se descubrió que en casi la totalidad de los casos la votación de los diputados es en bloque. Es decir, siguen una disciplina partidaria donde todos los pertenecientes al mismo partido votan igual, sin importar la provincia a la que pertenezcan. Además dicha disciplina suele aplicarse también para los interbloques. \\

Entonces el voto de un diputado esta condicionado a la decisión del partido y no esta condicionado por la provincia a la que representa. Entonces para el armado de la transacción se consideró que no es necesario contar con el nombre del diputado, sino por como voto su partido esa ley. \\

Además como cada provincia esta representada por un subconjunto de diputados, proporcional a la población de dicha provincia, que pertenecen a un subconjunto de los partidos, se determino que una forma conveniente de representar la decisión de la provincia es tomando el valor de la mayoría de los votos de los diputados que la representan. \\

En base a esto se decidió construir la transacción con el siguiente formato: \\

Ley $|$ Estado $|$ Partido1 $|$...$|$ Partido32 $|$ Provincia1 $|$ ... $|$ Provincia22 \\

Donde el valor de cada campo columna puede tomar los siguientes valores: \\

Para \textbf{Ley} \\

El nombre de archivo csv que corresponde al dataset de un tratamiento de ley. \\

Para \textbf{Estado} \\

\begin{itemize}
\item $LEY APROBADA$: Si la ley fue aprobada.
\item $LEY RECHAZADA$: Si la ley no fue aprobada. \\
\end{itemize}

 Para \textbf{Partido X} \\

\begin{itemize}
\item $PARTIDO [AFIRMATIVO]$
\item $PARTIDO [NEGATIVO]$
\item $PARTIDO [ABSTENCION]$
\item $PARTIDO [AUSENTE]$ 
\item $PARTIDO [EMPATE]$ \\
\end{itemize}

Donde el valor se define siguiendo el criterio de las mayorías. Para la ley que presenta la transacción se contabilizan los votos de todos los diputados del partido
y se asigna el resultado de la mayoria. EMPATE se utiliza si no hay un criterio mayoritario. \\

Por ejemplo: \\

Partido1 que tiene 5 diputados representantes, 4 votan afirmativo y 1 ausente, al haber mas votos positivos el resultado es PARTIDO1 [AFIRMATIVO] \\

Partido2 que tiene 2 diputados representantes, 1 vota afirmativo y otro 1 ausente. El resultado es PARTIDO1[EMPATE] \\

Para \textbf{Provincia X} \\

\begin{itemize}
\item $PROVINCIA [AFIRMATIVO]$
\item $PROVINCIA [NEGATIVO]$
\item $PROVINCIA [ABSTENCION]$
\item $PROVINCIA [AUSENTE]$ 
\item $PROVINCIA [EMPATE]$ \\
\end{itemize}

Donde el valor se define siguiendo el criterio de las mayorías. Para la ley que presenta la transacción se contabilizan los votos de todos los diputados que pertenecen a esa provincia, sin considerar el partido y se asigna el resultado de la mayoria. EMPATE se utiliza si no hay un criterio mayoritario. \\

Por ejemplo: \\

Una provincia que tiene 2 partidos que la representan en el recinto. El Partido1 tiene 5 diputados, 4 votan afirmativo y 1 ausente. El Partido2 tiene 3 diputados que votan de forma negativa. El resultado es PARTIDO1 [AFIRMATIVO]. \\

Esta representación de transacción permite clusterizar los datos individuales de cada diputado evitando la redundancia de datos permitiendo obtener reglas generalizadas sobre los partidos y los interbloques a los que pertenecen. \\

El motivo por el que las ausencias son consideradas en el armado de la transacción es que la presencia o no de un bloque en ciertos casos puede ser parte del armado de una estrategia legislativa, especialmente en los partidos que no son un monobloque.

\section{Reglas}\label{desarrollo}

Para encontrar las reglas que determinan el comportamiento del set de transacciones se utilizó la implementación del algoritmo $Apriori$ del paquete aRules del lenguaje de programación $R$.\\

El objetivo es a partir de las reglas encontradas, describir el comportamiento que lleva a que una ley sea aprobada o rechazada, para facilitar la descripción se analizan por separado las reglas por separado para luego sobre el fin del trabajo, para luego sobre el fin del trabajo establecer las reglas sobre la sanción o rechazo de la ley.\\

En la sección de $Diputados$ se describen las reglas que siguen los diputados en relación al partido que pertenecen.\\

En la sección de $Partidos$ se describen las reglas que relacionan a un partido con otro. Es decir que partidos votan frecuentemente de la misma forma o de forma opuesta.\\

En la sección de $Interbloques$ se describen las reglas que relacionan a los partidos del mismo interbloque y luego las del interbloque en conjunto contra los demás partidos.\\

En la sección de $Provincias$ se describen las reglas que relacionan a las provincias entre sí.\\

En la sección de $Partidos y Provincias$ se describen las reglas que relacionan a los partidos con las provincias.\\

En la sección de $Sanciones de Ley$ se describen las reglas son las que llevan a la aprobación o rechazo de una ley.

\subsection{Diputados}

Del análisis exploratorio de los datos en crudo se pudo observar y determinar que un diputado perteneciente a un partido cualquiera vota de la misma forma que el resto de sus compañeros de partido. Este comportamiento se repite en 71 de las 72 sesiones analizadas.

Sin importar la provincia a la que represente, un diputado vota alineado al partido al que pertenece.

\subsection{Partidos}

\subsubsection{Entre dos Partidos}

\subsubsection{Entre Conjunto de Partidos}


\subsection{Interbloques}

\subsubsection{Partidos del mismo interbloque}

\subsubsection{Interbloque contra el resto de los partidos}

\subsection{Provincias}

\subsection{Partidos y Provincias}
\subsection{Sanciones de Ley}

\section{Conclusiones}

\section{Referencias Bibliográficas}

\end{document}
